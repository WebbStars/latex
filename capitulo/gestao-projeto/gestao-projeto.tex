% ---
% Capitulo da Metodologia de Gestão e Desenvolvimento do Projeto
% ---
  Devido às características do projeto e da equipe, optou-se por utilizar as práticas do framework ágil \gls{scrum}. 

\section{Organização da Equipe}
As atividades foram divididas de acordo com o conhecimento técnico e habilidade. O \autoref{papeis} indica os papéis de cada membro da equipe.

\begin{quadro}[H]
	\centering\footnotesize
    \ABNTEXfontereduzida
    \caption{Papéis}
    \label{papeis}
    \resizebox {1 \textwidth }{!}{
        \begin{tabular}{|l|c|c|c|c|c|c|c|}
            \hline
            \thead{Papéis} & \thead{Cezar} & \thead{Henrique} & \thead{Isabela} & \thead{João} & \thead{Jonas} & \thead{Rafael} & \thead{Stefany} \\
            \hline
            PO &        &         &  \circlemark     &        &         &     &      \\
            \hline
            SM & \circlemark       &          &       &         &      &     &\\
            \hline
            DT & \circlemark       & \circlemark        &  \circlemark     &  \circlemark      & \circlemark        & \circlemark    &  \circlemark \\   
            \hline
        \end{tabular}
    }
\fonte{Autores}
\end{quadro}

\begin{itemize}
    \item \textbf{\ac{po}} é a \textbf{Isabela Souza Duarte}, responsável por definir as \gls{histórias de usuário} e priorização do \emph{\gls{backlog}} da equipe.
    \item \textbf{\ac{sm}} é o \textbf{Cezar Godoy Nascimento}, responsável por auxiliar o \ac{po} e a cumprir prazos de \gls{sprints}, priorizar itens do \emph{\gls{product-backlog}} e condenar as \emph{\gls{dailys}}.
    \item \textbf{\ac{dt}} - Cada integrante é responsável por determinada atividade para o desenvolvimento do projeto, desde a idealização do projeto e implementação de toda arquitetura proposta, infraestrutura, \emph{\gls{back-end}} e \emph{\gls{front-end}} da aplicação, garantindo assim o cumprimento dos requisitos do projeto. 
    O \autoref{divisao-atividade} aborda as distribuições das atividades.
\end{itemize}


\begin{quadro}[H]
	\centering\footnotesize
    \ABNTEXfontereduzida
    \caption{Distribuição de Atividades}
    \label{divisao-atividade}
    \resizebox {1 \textwidth }{!}{
        \begin{tabular}{|l|c|c|c|c|c|c|c|}
            \hline
            \thead{Atividade} & \thead{Cezar} & \thead{Henrique} & \thead{Isabela} & \thead{João} & \thead{Jonas} & \thead{Rafael} & \thead{Stefany} \\
            \hline
            Back-End &    \circlemark    &   \circlemark       &       &  \circlemark       &         &     &      \\
            \hline
            Blog & \circlemark       &  \circlemark        & \circlemark      &  \circlemark       &  \circlemark       &  \circlemark    & \circlemark     \\
            \hline
            Quadro Kanban & \circlemark       &  \circlemark        & \circlemark      &  \circlemark       &  \circlemark       &  \circlemark    & \circlemark     \\
            \hline
            Backlog & \circlemark       &          &  \circlemark     &       &        &      &      \\
            \hline
            Banco de Dados &   \circlemark   &  \circlemark        &       &   \circlemark      &         &     &   \\
            \hline
            Documentação e Modelagem & \circlemark       &  \circlemark        & \circlemark      &  \circlemark       &  \circlemark       & \circlemark     &  \circlemark   \\
            \hline
            Front-End &        &          &  \circlemark     &         &         & \circlemark    & \circlemark   \\
            \hline
            Vídeos &       &          &       &         &   \circlemark      &  \circlemark    &     \\
            \hline
        \end{tabular}
    }
\fonte{Autores}
\end{quadro}

\section{Gestão do Tempo}
As cerimônias ocorrem durante as \gls{sprints}. É de suma importância a participação dos \ac{dt}. Pois, é necessário para termos uma visão melhor de como anda o andamento das atividades da equipe. Portanto, observe o tempo de cada cerimônia abaixo:
\begin{itemize}
    \item \textbf{Sprint:} Duração de 7 dias;
    \item \textbf{Planejamento da Sprint:} Duração de 2 horas;
    \item \textbf{\emph{Daily} Scrum:} Duração de 15 minutos;
    \item \textbf{\emph{Review:}} Duração de 30 minutos;
    \item \textbf{Retrospectiva da Sprint:}  Duração de 1 hora;
\end{itemize}
\subsection{Sprints}
O andamento das tarefas é acompanhado nas \emph{\gls{dailys}} e no \emph{\gls{board}} do GitHub Projects onde é organizado e priorizado as atividades do projeto. O \autoref{andamento-tarefas} mostra todas as \gls{sprints} e seu determinado objetivo.

\begin{quadro}[H]
\centering\footnotesize
\caption{Sprints}
\label{andamento-tarefas}
        \begin{tabular}{|p{0.25\linewidth} | p{0.10\linewidth} | p{0.60\linewidth} |}  \hline
        \thead{Sprints} & \thead{Duração} & \thead{Objetivo} \\
        \hline
        Sprint 1 - 09/08 ao 15/08 & 7 dias &   Definição da proposta, criação do blog, canal no Youtube, github e \ac{svn}.
        \\
        \hline
        Sprint 2 - 16/08 ao 22/08 & 7 dias & Estruturação da documentação inicial e apresentação da proposta. 
        \\
        \hline
        Sprint 3 - 23/08 ao 29/08 & 7 dias & Elicitação de \gls{requisitos-funcionais}, \gls{requisitos-nao-funcionais}, \gls{histórias de usuário} e \gls{modelagem}.
        \\
        \hline
        Sprint 4 - 30/08 ao 05/09 & 7 dias & Correção da documentação baseada no feedback dos professores.  
        \\
        \hline
        Sprint 5 - 06/09 ao 12/09 & 7 dias & Finalização de ajustes na documentação e entrega do desenho da aplicação.  
        \\
        \hline
        Sprint 6 - 13/09 ao 19/09 & 7 dias & Ajustes na documentação do desenho da aplicação solicitados pelos professores e continuação no desenvolvimento do \ac{der} e o \ac{mer}.  
        \\
        \hline
        Sprint 7 - 20/09 ao 26/09 & 7 dias & Desenvolvimento e entrega da apresentação do desenho da aplicação.  
        \\
        \hline
        Sprint 8 - 27/09 ao 03/10 & 7 dias & Inicialização da \ac{poc}, criação do \gls{front-end} e \gls{back-end}.  
        \\
        \hline
        Sprint 9 - 04/10 ao 10/10 & 7 dias & Andamento do desenvolvimento da \ac{poc}.  
        \\
        \hline
        Sprint 10 - 11/10 ao 17/10 & 7 dias & Finalização e entrega da \ac{poc}.  
        \\
        \hline
        Sprint 11 - 18/10 ao 25/10 & 7 dias & Finalização da documentação e desenvolvimento do \ac{mvp}. 
        \\
        \hline
        Sprint 12 - 26/10 ao 01/11 & 7 dias & Finalização e entrega da documentação e desenvolvimento do \ac{mvp}. 
        \\
        \hline
        Sprint 13 - 02/11 ao 08/11 & 7 dias & Finalização e entrega da documentação e desenvolvimento do \ac{mvp}. 
        \\
        \hline
        Sprint 14 - 09/11 ao 15/11 & 7 dias & Realização de melhorias do \ac{mvp}. 
        \\
        \hline
        Sprint 15 - 16/11 ao 22/11 & 7 dias & Apresentação do \ac{mvp}. 
        \\
        \hline
        Sprint 16 - 23/11 ao 29/11 & 7 dias & Apresentação do \ac{mvp}. 
        \\
        \hline
        Sprint 17 - 30/11 ao 06/12 & 7 dias & Realizar ajustes de entregar versão final da documentação  do \ac{mvp}. 
        \\
        
        
        \hline
        \end{tabular}
\fonte{Autores}
\end{quadro}

\section{Ferramenta de Gerenciamento do Projeto}
\begin{itemize}
    \item \textbf{GitHub Projects:} Foi utilizada a ferramenta do GitHub Projects, o qual destina-se para o gerenciamento de equipes, o qual permite listar as tarefas que devemos realizar na semana e consequentemente da visualização do status de cada tarefa no \emph{\gls{board}}.
\end{itemize}

\section{Ferramentas para Comunicação da Equipe}
\begin{itemize}

    \item \textbf{Whatsapp e Google Meet:} As ferramentas foram utilizadas com o propósito de realizar as reuniões da equipe e alinhamento das atividades.
\end{itemize}