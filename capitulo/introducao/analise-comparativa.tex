\section{Análise Comparativa}
    O mercado para o serviço de au pair exige qualificações específicas, pois trata-se de uma atividade sensível e que depende diretamente do índice de intercâmbios bem-sucedidos.
    
    As agências de au pair se destacam pelo índice de aprovação e de confiança das experiências realizadas. 
    Notamos que existem no mercado dois caminhos para a construção de uma boa experiência de au pair.
    
    A primeira delas, seria por meio de interface de uma agência especializada, pois são elas que analisam os perfis tanto das famílias anfitriãs quanto das au pairs. 
    
    Neste caso, a análise de mercado e de concorrência dessa atividade está diretamente ligada ao desempenho das agências e da qualidade das experiências au pairs, das quais se destaca no mercado a agência
    \cite{auPairInAmerica2022}.
    
    Reconhecida como a mais antiga e respeitada agência de au pairs, essa agência será responsável por todo o processo de cadastro, análise, avaliações, documentações, conexões e acompanhamento do intercâmbio.
    
    Numa tradução livre o site da agência \cite{auPairInAmerica2022} diz que “por mais de 30 anos, oferece as melhores oportunidades culturais de cuidados infantis para famílias anfitriãs em todo o país \ac{usa} e au pairs de todo o mundo. Existem outras agências de au pair, mas au pair é o primeiro programa de au pair legal do país, designado pelo governo dos \ac{usa} em 1986”.
    
    O diferencial desta agência está em ser reconhecida pelo governo americano, o que obviamente contribui com a legalidade do serviço, já que parte das atividades do au pair é trabalhar e ser remunerado dentro dos padrões e das leis americanas, aspecto este, que em outros formatos pode trazer complicações, especialmente em relação a questão do visto, que neste caso é todo arranjado pela agência. Obviamente, este serviço de agência de au pair tem um custo ao candidato.
    
    Outro caminho para a construção de uma experiência de intercâmbio au pair, é o oferecido pela plataforma de serviço de \gls{match} entre famílias anfitriãs e au pairs, chamado de
    \cite{auPair.com2022}, trata-se de um site de conexão entre famílias e candidatos au pair diretamente. 
    
    O site da aupair.com diz que ”\cite{auPair.com2022} é uma agência de correspondência online - que é segura e fácil de usar. Como uma plataforma de correspondência online, não podemos participar diretamente do processo de arranjo de au pair, pois não somos uma agência de serviço completo. No entanto, podemos ajudá-lo com informações úteis em nosso site. Se você não quiser organizar sua estadia como au pair, entre em contato com uma de nossas agências de au pair parceiras confiáveis, que irá guiá-lo durante todo o processo e sua estadia como au pair!”
    
    Ou seja, nesta plataforma todos os arranjos são feitos pelos próprios usuários, o site só disponibiliza a possibilidade do próprio au pair realizar todo o processo de encontro da família anfitriã, ou da família anfitriã encontrar diretamente o au pair, sem a interferência de uma agência. Desta forma, todos os trâmites e negociações serão acordados diretamente entre os interessados.
    
    Este serviço também é legalizado, porém, neste formato pode haver situações fora dos padrões e dos controles de qualidade que o serviço de au pair mediado por uma agência oferece e garante. Por exemplo, acordos de remuneração, horas trabalhadas e acompanhamento das atividades durante o intercâmbio. Esta plataforma de \gls{match} au pair é gratuita.
    
    Mesmo com essas diferenças de formatos, estas duas ofertas se destacam no mercado de au pair pela sua atuação e por conta do número de intercâmbios realizados e bem sucedidos, ambas divulgam seus serviços em inglês e por meio de site na internet, os quais exigem inscrição de famílias anfitriãs e au pairs a partir de uma ampla etapa de envio e análise de documentos, entrevistas e qualificações. Essas características dos serviços de ambos os sites transmitem a seriedade e compromisso para que a segurança dos usuários, famílias anfitriãs e candidatos au pair, estejam em primeiro lugar. 
    
    Ambas possuem representação no Brasil, respectivamente pela \cite{experimCultl2022} e pela \cite{centralInter2022}. As duas plataformas disponibilizam potentes possibilidades de encontro e realização de intercâmbio au pair, porém existem características que diferem o formato que o processo será realizado em cada uma delas, e é este o diferencial de cada ferramenta. A possibilidade da autonomia proposta em um deles é interessante, mas expõe os usuários a uma experiência fora dos padrões, pois por ser concretizado os acordos diretamente entre os interessados, não há manutenção externa ou controle de qualidade.
    
    Segundo reportagem da \ac{bbc}, este serviço tem sido interpretado como a “escravidão moderna”, pois há um disparate em relação aos valores e acordos que tem sido firmados e legislação trabalhista vigente \cite{bbcNewsBrasil2017}.
    
    Já no formato onde a agência coordena as tramitações, além do custo, os interessados estarão sujeitos às exigências e acordos pré-estabelecidos, porém é dessa forma que a alta qualidade do encontro de au pairs e famílias anfitriãs será garantido.
    
    A proposta de aplicação web terá uma atuação diferenciada dos serviços oferecidos nestes dois formatos apresentados, pois a plataforma tem como objetivo ampliar as ofertas de todos os sujeitos envolvidos no contexto au pair, quais sejam: famílias anfitriãs, agências e candidatos au pairs.
    
    O \autoref{quadro-analise-de-mercado} a seguir ilustra nossa análise da concorrência:
        \begin{quadro}[H]
        \caption{Análise da Concorrência}
        \label{quadro-analise-de-mercado}
        \small
        \begin{tabular}{|l|c|c|c|}
        \hline
        \textbf{Funcionalidades} & \multicolumn{1}{l|}{\textbf{Au Pair in América}} & \multicolumn{1}{l|}{\textbf{AuPair.com}} & \multicolumn{1}{l|}{\textbf{AupaMatch}} \\ \hline
        
        Gerenciador de usuário               & \circlemark & \circlemark & \circlemark \\ \hline
        Identificar e autenticar com usuário & \circlemark & \circlemark & \circlemark \\ \hline
        Definir perfil de usuário            & \circlemark & \circlemark & \circlemark \\ \hline
        Gerenciador de anúncio de vaga       &   &   & \circlemark \\ \hline
        Aplicar candidatura em vagas         &   &   & \circlemark \\ \hline
        Salvar vagas em favoritos            &   & \circlemark & \circlemark \\ \hline
        Consolidar vagas                     &   &   & \circlemark \\ \hline
        Filtro de busca                      &   &   & \circlemark \\ \hline
        Priorização e vagas                  &   &   & \circlemark \\ \hline
        \end{tabular}
        \fonte{Autores}
        \end{quadro}
