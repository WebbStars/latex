% ---
% Capitulo da Introdução
% ---
    O intercâmbio au pair é um programa que oferece a chance de estudar em um país estrangeiro e morar com uma família anfitriã, trabalhando legalmente cuidando das crianças dessa família e recebendo um salário semanal para realizar este serviço. O processo para se tornar au pair tem como base se adequar aos pré-requisitos, como ter determinada idade, possuir experiência comprovada com crianças, entre outros aspectos. Além de preencher os pré-requisitos, em alguns casos se faz necessário passar por uma análise de perfil, que é realizada por uma agência especializada.

    A agência tem o papel de guiar a candidata au pair durante o processo de escolha e aceitação da vaga. Uma au pair pode se cadastrar em diversas agências e tentar se conectar com muitas famílias ao procurar uma vaga na qual o seu perfil se encaixe. Porém, já é possível que a família e a au pair tenham contato através de redes sociais, indicação de outras famílias, entre outros.

    As agências tentam atrair au pairs, destacando os aspectos positivos da experiência como sendo um intercâmbio cultural barato e vantajoso. Segundo anúncio no site Experimento Intercâmbio.
    
    \begin{citacao}
        Se você tem entre 18 e 26 anos, é solteira, gosta de crianças, tem carteira de habilitação, concluiu o ensino médio e tem nível de inglês intermediário, este é o intercâmbio para você. Você será recebida como um membro da família e terá a oportunidade de fazer um curso a sua escolha, enquanto cuida das crianças, vivendo uma nova cultura e praticando o inglês todos os dias! Durante sua estadia nos Estados Unidos, você receberá uma bolsa de estudos e um salário semanal por seu trabalho como au pair. O salário mínimo semanal é calculado pelo Departamento de Estado Americano considerando os custos de moradia e alimentação. As famílias anfitriãs e au pairs são livres para combinar uma remuneração maior do que o mínimo legalmente estabelecido. Além disso, você terá um dia e meio de folga por semana de trabalho e um final de semana livre por mês. Além de 2 semanas de férias no ano. Um sonho essa experiência!
        \cite{experimCultl2022}
    \end{citacao}

    É interessante destacar que o programa é oferecido a jovens que atendam aos pré-requisitos, porém, a maioria das pessoas que realizam este intercâmbio são do sexo feminino, e por isso, em muitos locais encontra-se inclusive a expressão “a au pair” e em muitos sites a linguagem é toda direcionada às mulheres. Por isso, caso o candidato seja do sexo masculino, é importante que ele esteja atento se há essa exigência na vaga.
    
    \begin{citacao}
        O intercâmbio de au pair para homens não é a modalidade mais comum. Por conta disso, nem todas as agências oferecem vagas para homens como au pair no Exterior. Então, antes de mais nada, o candidato tem que procurar alguma agência que ofereça a modalidade deste intercâmbio também para homens. A verdade é que cada vez mais famílias optam por au pair homens para tomar conta de seus filhos.
        \cite{partiuIntercam2022}
    \end{citacao}
    
    As regras de au pair no geral são as mesmas para homens e para mulheres, segundo o site \cite{partiuIntercam2022}, os candidatos e candidatas devem ter inglês entre intermediário e avançado (e isso será comprovado mediante entrevista), deverão ter carteira de habilitação para dirigir carro, ter terminado o ensino médio e disponibilidade de fazer o programa por um ano. Porém, a idade recomendada para as mulheres é entre 18 e 26 anos e para homens, entre de 18 a 27 anos, os homens também precisam comprovar horas de trabalho com crianças (creches, escolas, orfanatos, projetos sociais, filhos de vizinhos, etc), já as mulheres nem sempre há essa exigência, gostar de criança em alguns casos já é suficiente para se adequar aos pré requisitos, e por fim ambos não podem ter filhos.

    O processo para se tornar au pair pode ser cansativo, uma vez que a mesma deve estar sempre atenta a novas publicações de vagas, estando assim à mercê da plataforma da agência e da procura por au pairs por parte das famílias.

    Pensando principalmente na dificuldade de comunicação que esse processo pode acarretar, e em como isso seria prejudicial a todos os envolvidos (au pairs, famílias e agências), pode-se observar que a criação de uma plataforma que visa principalmente a facilitação da comunicação e um consolidado de vagas em um único lugar se faz necessária. 

    A partir dessa observação, o seguinte projeto em desenvolvimento do \href{https://svn.spo.ifsp.edu.br/svn/a6pgp/S202202-PI-NOT/WebbStars}{Grupo 01} (\href{https://webbstarsifspgrupo1.blogspot.com/}{WebbStars}) se propõe a construir uma plataforma que lide com os problemas expostos anteriormente, além de adicionar outras funcionalidades.
