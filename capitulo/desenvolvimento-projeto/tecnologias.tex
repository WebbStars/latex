\graphicspath{./anexos}
\section{Tecnologias utilizadas}

\subsection{Front-end}

Se fez necessário a criação de um cliente \gls{spa}, com a biblioteca \ac{React}, visando um menor consumo de recursos do \ac{back-end} e maior dinamismo na experiência de usuário, para além do aproveitamento de código, graças aos componentes que podem ser criados por meio dela. Acrescenta-se o uso do \ac{TypeScript}, objetivando a prevenção de erros de tipagem. A estilização se deu graças ao framework \ac{material}, que fornece classes \ac{css} combinadas para customização de elementos \ac{html}.

\subsection{Back-end}
Para o desenvolvimento do \ac{back-end} do projeto foi utilizado o \ac{node}, o qual foi
escolhido para abarcar as regras de negócio, a migração do banco de dados e fornecimento
da \ac{api}. O banco de dados relacional inicial eleito foi o \ac{MySQL}, mas os membros da equipe possuem mais familiaridade com o \ac{mongodb}. Por isso, optou-se de realizar essa troca.

Por fim, temos o \ac{Nginx} que é usado como proxy reverso, a fim de controlar quais portas e arquivos são disponibilizadas publicamente. 

\subsection{Versionamento, integração contínua e deployment}
O versionamento foi realizado por via da ferramenta \ac{GIT}, e da plataforma de
hospedagem \ac{GitHub}, seguindo o fluxo de trabalho \ac{GitHub Flow}. Através do \ac{GitHub
Actions}, foi feita a execução da integração contínua e montagem (build) da imagem Docker
da aplicação \ac{node}. Inicialmente, foi escolhido o \ac{aws} para as imagens serem carregadas e executada no ambiente, além dos arquivos
estáticos. Mas, como o AWS não possui conexão com o \ac{mongodb}, elegeu-se o \ac{render} para o deploy.