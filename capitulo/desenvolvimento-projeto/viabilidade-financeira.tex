\section{Viabilidade Financeira}
A análise de viabilidade financeira é importante para qualquer negócio, porque mostra se é possível para uma empresa crescer, quais os riscos envolvidos e se os lucros esperados superam os custos fixos e variáveis.

Em um projeto de desenvolvimento de software, os custos podem ser divididos em dois grandes grupos: os custos fixos e os custos variáveis. Os custos fixos são aqueles que não variam em função do tamanho do projeto, por exemplo, despesas com arrendamento de espaço físico, com aluguel de equipamentos e com mão de obra fixa. Os custos variáveis são aqueles que dependem do tamanho do projeto, como custos com materiais, com mão de obra variável e as despesas com pessoas que trabalham fora do ambiente de desenvolvimento.

\subsection{Análise de Mercado}

Neste cenário de transições digitais onde tudo foi ou vai para internet, entendemos que surge a necessidade de uma ferramenta tecnológica de conexão e intermediação entre au pairs, famílias anfitriãs e as agências de intercâmbio au pair, para gerenciar aspectos no que diz respeito às atividades deste público. A plataforma será desenvolvida para atender a essa necessidade, possibilitando descentralização, promovendo mais transparência, bem como planejamento e ajuste nos objetivos dos envolvidos. 
Dessa forma, o AupaMatch, além de contribuir com os interessados, gerencia as informações relacionadas.