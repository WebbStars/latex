% Definições para glossario

% ATENCAO o SHARELATEX GERA O GLOSSARIO/LISTAS DE SIGLAS SOMENTE UMA VEZ
% CASO SEJA FEITA ALGUMA ALTERAÇÃO NA LISTA DE SIGLAS OU GLOSSARIO É NECESSARIO UTILIZAR A OPÇÃO :
% "Clear Cached Files" DISPONIVEL NA VISUALIZAÇÃO DOS LOGS 
% ---
% https://www.sharelatex.com/learn/Glossaries

% Normalmente somente as palavras referenciadas são impressas no glossário, portanto é necessário referenciar utilizando :
% \gls{identificação}            
% \Gls{identificação}            
% \glspl{identificação}            
% \Glspl{identificação}    

\newglossaryentry{pai}{
                name={pai},
                plural={pai},
                description={este é uma entrada pai, que possui outras
                subentradas.} }

 \newglossaryentry{componente}{
                name={componente},
                plural={componentes},
                parent=pai,
                description={descriação da entrada componente.} }
 
 \newglossaryentry{filho}{
                name={filho},
                plural={filhos},
                parent=pai,
                description={isto é uma entrada filha da entrada de nome
                \gls{pai}. Trata-se de uma entrada irmã da entrada
                \gls{componente}.} }



\newglossaryentry{crud} {
    name=CRUD,
    plural= {CRUDs},
    description={Create Retrieve Update Delete - Interface de usuário para manutenção em
    banco de dados que consiste somente nas operações básicas do banco de dados, sem nenhum
    tipo de inteligencia adicional de forma a facilitar ao usuário com o tratamento de algum processo}
}

\newglossaryentry{jpg} {
    name=JPG,
    description={Imagem em formato JPEG}
}

\newglossaryentry{moodle} {
    name=Moodle,
    description={Plataforma para auxiliar na organização de disciplinas de forma online,
    no IFSP SPO disponível no endereço \url{https://eadcampus.spo.ifsp.edu.br}}
}

\newglossaryentry{tag} {
    name=tag,
    plural= {tags},
    description={Um marcador, a palavra em inglês significa etiqueta}
}
                
\newglossaryentry{whatsapp} {
    name=Whatsapp,
    description={Aplicativo de mensagens instantâneas que também permite mensagens de texto,
    voz e conferencias de voz e vídeo}
}

\newglossaryentry{regras-negocio}{
    name=Regras de Negócio,
    description={Servem para definir regras básicas que o projeto deve respeitar}
}

\newglossaryentry{sprint-planning}{
    name=Sprint Planning,
    description={É definido como um evento que abre um novo ciclo de execução, a Sprint}
}

\newglossaryentry{scrum}{
    name=Scrum,
    description={Scrum é uma estrutura para organizar as demandas e executar as tarefas,
    permitindo uma entrega rápida e de alta qualidade do produto}
}

\newglossaryentry{kanban}{
    name=Kanban,
    description={O kanban trata-se de um sistema visual que busca gerenciar o
    trabalho conforme ele se move pelo processo}
}

\newglossaryentry{back-end}{
    name=Back-end,
    description={O back-end se relaciona com o que está por trás das aplicações
    desenvolvidas na programação}
}

\newglossaryentry{front-end}{
    name=Front-end,
    description={O front-end está muito relacionado com a interface gráfica do projeto}
}

\newglossaryentry{sprints}{
    name=Sprints,
    description={O sprint é uma ferramenta para atingir objetivos em um determinado período de tempo}
}

\newglossaryentry{dailys}{
    name=Dailys,
    description={A daily é um reunião diária de 15 minutos que ocorre durante o Sprint,
    para trazer alinhamento e comprometimento do time de desenvolvimento}
}

\newglossaryentry{board}{
    name=Board,
    description={É um painel do Kanban físico ou digital de gerenciamento de projeto
    que auxilia na visualização de trabalho}
}

\newglossaryentry{product-backlog}{
    name=Product Backlog,
    description={O Product Backlog é uma lista ordenada de todos os requisitos que
    se tem conhecimento de que precisam estar no produto}
}

\newglossaryentry{backlog}{
    name=Backlog,
    description={Sprint Backlog é uma lista de tarefas que o time se compromete a fazer em um Sprint}
}

\newglossaryentry{histórias de usuário}{
    name=Histórias de Usuário,
    description={Uma história do usuário é uma explicação informal e geral sobre um recurso
    de software escrita a partir da perspectiva do usuário final}
}

\newglossaryentry{requisitos-funcionais}{
    name=Requisito Funcional,
    description={Toda abstração de um recurso, funcionalidade ou resultado esperado de um sistema.}
}

\newglossaryentry{requisitos-nao-funcionais}{
    name=Requisito Não Funcional,
    description={Trata-se de algo que não é uma funcionalidade, mas que precisa 
    ser realizado para que o software atenda seu propósito.}
}

\newglossaryentry {diagrama de classes} {
    name=Diagrama de classes,
    description={Representa os tipos (classes) de objetos de um sistema}
}

\newglossaryentry{modelo-logico}{
    name=Modelo Lógico,
    description={Estabelece a estrutura dos elementos de dados e os relacionamentos entre eles. 
    É independente do banco de dados físico que detalha como os dados serão implementados. 
    O modelo lógico serve como um modelo para os dados usados.}
}

\newglossaryentry{JavaScript}{
    name=JavaScript,
    description={TJavaScript é uma linguagem de programação interpretada estruturada, de script em alto nível com tipagem dinâmica fraca e multiparadigma. Juntamente com HTML e CSS, o JavaScript é uma das três principais tecnologias da World Wide Web.}
}

\newglossaryentry{React}{
    name=React,
    description={O React é uma biblioteca JavaScript de código aberto com foco em
    criar interfaces de usuário em páginas web.}
}

\newglossaryentry{TypeScript}{
    name=TypeScript,
    description={TypeScript é uma linguagem de programação de código aberto desenvolvida pela Microsoft. 
    É um superconjunto sintático estrito de JavaScript e adiciona tipagem estática opcional à linguagem.}
}

\newglossaryentry{Tailwind CSS}{
    name=Tailwind CSS,
    description={Tailwind CSS é uma estrutura CSS de código aberto. A principal característica desta
    biblioteca é que, ao contrário de outros frameworks CSS como Bootstrap,
    ela não fornece uma série de classes predefinidas para elementos como botões ou tabelas.}
}

\newglossaryentry{Python Django}{
    name=Python Django,
    description={Django é um framework para desenvolvimento rápido para web,
    escrito em Python, que utiliza o padrão model-template-view.}
}

\newglossaryentry{Django REST Framework}{
    name=Django REST,
    description={Django REST Framework ou DRF é uma biblioteca que permite a construção de
    APIs REST utilizando a estrutura do Django.}
}

\newglossaryentry{PostgreSQL}{
    name=PostgreSQL,
    description={PostgreSQL é um sistema gerenciador de banco de dados objeto relacional,
    desenvolvido como projeto de código aberto.}
}

\newglossaryentry{Nginx}{
    name=Nginx,
    description={Nginx é um servidor leve de HTTP, proxy reverso, proxy de e-mail IMAP/POP3.}
}

\newglossaryentry{GitHub}{
    name=GitHub,
    description={Plataforma de hospedagem de código-fonte e arquivos com controle de versão usando o Git.}
}

\newglossaryentry{GitHub Flow}{
    name=GitHub Flow,
    description={O fluxo do GitHub é um fluxo de trabalho leve e baseado em ramificação. O fluxo do
    GitHub pode ser utilizado para política, documentação e roteiro do site.}
}

\newglossaryentry{GitHub Actions}{
    name=GitHub Actions,
    description={O GitHub Actions é uma plataforma de integração contínua e entrega contínua (CI/CD)
    que permite automatizar seu pipeline de compilação, teste e implantação.}
}

\newglossaryentry{Docker}{
    name=Docker,
    description={Docker é um conjunto de produtos de plataforma como serviço que usam
    virtualização de nível de sistema operacional para entregar software em pacotes chamados contêineres.}
}

\newglossaryentry{Python}{
    name=Python,
    description={Python é uma linguagem de programação de alto nível, interpretada de script,
    imperativa, orientada a objetos, funcional, de tipagem dinâmica e forte.}
}

\newglossaryentry{Modelo Conceitual}{
    name=Modelo Conceitual,
    description={Mostra todos os conceitos importantes do domínio do sistema, bem como as
    associações entre esses conceitos.}
}

\newglossaryentry{Cross-Origin Resource Sharing (CORS)}{
    name=CORS,
    description={Significa Cross-Origin Resource Sharing, em português (Compartilhamento de
    recursos com origens diferentes) é um mecanismo que usa cabeçalhos adicionais HTTP para
    informar a um navegador que permita que um aplicativo Web seja executado em uma origem
    (domínio) com permissão para acessar recursos selecionados de um servidor em uma origem distinta.}
}

\newglossaryentry{HTTP Post}{
    name=HTTP Post,
    description={Envia dados ao servidor, o tipo do corpo da solicitação é indicado pelo cabeçalho
    Content-Type.}
}

\newglossaryentry{HTTP}{
    name=HTTP,
    description={Do inglês Hypertext Transfer Protocol. Sendo um protocolo de transferência que possibilita 
    que as pessoas que inserem a URL do seu site na Web possam ver os conteúdos e dados que nele existem.}
}

\newglossaryentry{HTTPS}{
    name=HTTPS,
    description={Do inglês Hyper Text Transfer Protocol Secure, esse protocolo é a combinação dos
    protocolos HTTP e SSL (Secure Sockest Layers). Considerado o mais seguro é
    porque ele faz a encriptação dos dados fornecidos.}
}

\newglossaryentry{SSL}{
    name=SSL,
    description={Protocolo de criptografia de dados em trânsito na versão 3.0 do SSL exige a
    autenticação de ambas as partes envolvidas na troca de mensagens.}
}

\newglossaryentry{Timestamp}{
    name=Timestamp,
    description={Em português (estampa de tempo) é uma cadeia de caracteres denotando a
    hora ou data que certo evento ocorreu.}
}
\newglossaryentry{TLS}{
    name=TLS,
    description={É a sigla de Transport Layer Security, ou seja, um protocolo de segurança
    cuja finalidade é facilitar a segurança e privacidade de dados de dados na internet.}
}

\newglossaryentry{ROI}{
    name=ROI,
    description={significa “Return over Investment” ou “Retorno sobre Investimento”, sendo 
    uma métrica financeira baseada na relação entre o dinheiro ganho e o dinheiro aplicado em um investimento.}
}

\newglossaryentry{LGPD}{
    name=LGPD,
    description={A Lei Geral de Proteção de Dados Pessoais (LGPD), Lei n° 13.709/2018, foi promulgada
    para proteger os direitos fundamentais de liberdade e de privacidade e a livre
    formação da personalidade de cada indivíduo.}
}

\newglossaryentry{logs}{
    name=Logs,
    description={Em português registro, são arquivos de texto nos quais são registradas
    informações diversas, geralmente seguindo a cronologia dos eventos acontecidos.}
}

\newglossaryentry{cookies}{
    name=Cookies,
    description={É um pedaço de texto que um servidor Web pode armazenar no disco rígido do usuário.
    São utilizados pelos sites principalmente para identificar e armazenar informações sobre os visitantes.}
}

\newglossaryentry{HTTP Only}{
    name=HTTP Only,
    description={É uma bandeira que o site pode especificar sobre um cookie.}
}

\newglossaryentry{cross site scripting (XSS)}{
    name=XSS,
    description={Significa “Cross site scripting”, é um tipo de vulnerabilidade de sites por meio da
    qual um atacante é capaz de inserir scripts maliciosos em páginas e aplicativos que seriam
    confiáveis e usá-los para instalar malwares nos navegadores dos usuários. }
}

\newglossaryentry{frameworks}{
    name=Frameworks,
    description={é uma estrutura de códigos genérica que tem o objetivo de prover uma nova função dentro do código.}
}

\newglossaryentry{whitelist}{
    name=Whitelist,
    description={Em português "lista branca", também chamada de ‘lista do bem’, formada por um conjunto
    de e-mails, domínios ou endereços IP, previamente aprovados e com permissão de entrega, sem a
    necessidade de serem submetidos a filtros}
}

\newglossaryentry{modelagem}{
    name=Modelagem de BD,
    description={Modelagem de Banco de Dados, é a etapa de obter uma demonstração de como serão estruturados os dados.}
}

\newglossaryentry{timestamp}{
    name=Timestamp,
    description={Representa um ponto específico na linha do tempo e leva em consideração o fuso
    horário em questão (UTC).}
}

\newglossaryentry{OWASP}{
    name=OWASP,
    description={O OWASP (Open Web Application Security Project), ou Projeto Aberto de Segurança em Aplicações Web, é uma comunidade online que cria e disponibiliza de forma gratuita artigos, metodologias, documentação, ferramentas e tecnologias no campo da segurança de aplicações web.}
}

\newglossaryentry{phishing}{
    name=Phishing,
    description={É uma técnica de engenharia social usada para enganar usuários de internet usando fraude eletrônica para obter informações confidenciais, como nome de usuário, senha e detalhes do cartão de crédito.}
}

\newglossaryentry{malware}{
    name=Malware,
    description={É qualquer software intencionalmente feito para causar danos a um computador, servidor, cliente, ou a uma rede de computadores.}
}

\newglossaryentry{XML}{
    name=XML,
    description={eXtensible Markup Language, é uma linguagem de marcação recomendada pela W3C para a criação de documentos com dados organizados hierarquicamente, tais como textos, banco de dados ou desenhos vetoriais.}
}

\newglossaryentry{URL}{
    name=URL,
    description={O Uniform Resource Locator, se refere ao endereço de rede no qual se encontra algum recurso informático, como um arquivo de computador ou um dispositivo periférico (impressora, equipamento multifuncional, unidade de rede etc.). Essa rede pode ser a Internet, uma rede corporativa (como uma intranet).}
}

\newglossaryentry{match}{
    name=Match,
    description={A palavra match é de origem inglesa e significa combinação. Portanto, quando se dá o match no app, quer dizer que duas pessoas possuem afinidades, podendo evoluir de um mero contato para um relacionamento.}
}

\newglossaryentry{high-school}{
    name=high school,
    description={O programa high school é direcionado para adolescentes e o significado de High School é colegial, ensino médio. O High School é a fase escolar que compreende o nosso ensino médio aqui no Brasil.}
}

\newglossaryentry{GIT}{
    name=GIT,
    description={É um sistema de controle de versão open-source.}
}

\newglossaryentry{feedbacks}{
    name=feedbacks,
    description={Feedback, palavra originária da língua inglesa, significa opinião, retorno, avaliação ou comentário. Na prática, é também um termo incorporado ao idioma português, sendo empregado justamente para expressar um ponto de vista.}
}

\newglossaryentry{startups}{
    name=startups,
    description={Uma startup é um grupo de pessoas à procura de um modelo de negócios repetível e escalável, trabalhando em condições de extrema incerteza.}
}

\newglossaryentry{proof-of-concept}{
    name=Proof of Concept,
    description={Proof of Concept (POC ou Prova de Conceito) é um termo aplicável tanto às empresas ou startups de tecnologia como ao desenvolvimento de software. No primeiro caso, ao olharmos para o mercado, a POC tem o objetivo de validar um produto, considerando seus custos e um público específico.}
}

\newglossaryentry{minimum-viable-product}{
    name=Minimum Viable Product,
    description={O MVP ou Minimum Viable Product é um produto com recursos suficientes para atrair clientes pioneiros e validar uma ideia de produto no início do ciclo de desenvolvimento do produto . Em setores como software, o MVP pode ajudar a equipe de produto a receber feedback do usuário o mais rápido possível para iterar e melhorar o produto}
}

\newglossaryentry{facebook}{
    name=Facebook,
    description={O Facebook é uma rede social que permite conversar com amigos e compartilhar mensagens, links, vídeos e fotografias.}
}

\newglossaryentry{apple}{
    name=Apple,
    description={Apple Inc. é uma empresa multinacional norte-americana que tem o objetivo de projetar e comercializar produtos eletrônicos de consumo, software de computador e computadores pessoais}
}

\newglossaryentry{dropbox}{
    name=Dropbox,
    description={Dropbox é um serviço para armazenamento e partilha de arquivos. É baseado no conceito de "computação em nuvem". Ele pertence ao Dropbox Inc., sediada em San Francisco, Califórnia, EUA. A empresa desenvolvedora do programa disponibiliza centrais de computadores que armazenam os arquivos de seus clientes.}
}

\newglossaryentry{MySQL}{
    name=MySQL,
    description={MySQL é um sistema de gerenciamento de banco de dados (SGBD), que utiliza a linguagem SQL (Linguagem de Consulta Estruturada, do inglês Structured Query Language) como interface. }
}

\newglossaryentry{node}{
    name=Node.js,
    description={É um ambiente de execução JavaScript que permite executar aplicações desenvolvidas com a linguagem de forma autônoma, sem depender de um navegador.}
}

\newglossaryentry{mongodb}{
    name=MongoDB,
    description={É um banco de dados orientado a documentos que possui código aberto.}
}

\newglossaryentry{render}{
    name=Render,
    description={É uma plataforma unificada para criar e executar todos os seus aplicativos e sites com SSL grátis, CDN global, redes privadas e implementações automáticas do Git.}
}

\newglossaryentry{material}{
    name=Material UI,
    description={Material UI é uma estrutura front-end de código aberto para componentes React.}
}