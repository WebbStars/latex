% ---
% RESUMOS
% ---

% resumo em português
\setlength{\absparsep}{18pt} % ajusta o espaçamento dos parágrafos do resumo
\begin{resumo}
\todo[inline]{fazer o seu resumo, ele só é feito depois que o documento está terminado
\newline
\newline os itens em negrito estão aqui para ressaltar detalhes que devem ser seguidos, mas não se utiliza o negrito em um resumo}

 De acordo com a norma \citetitle{NBR6028:2003} (3.1-3.2) \index{NBR6028}, o resumo\index{resumo} deve ressaltar o contexto, o objetivo, o método, os resultados e as conclusões do documento (portanto deve ser escrito por ultimo). A ordem e a extensão destes itens dependem do tipo de resumo (informativo ou indicativo) e do  tratamento que cada item recebe no documento original. O resumo \textbf{deve ter um paragrafo único} e deve \textbf{ter entre 150 e 500 palavras para trabalhos acadêmicos ou entre 100 e 250 para artigos de periódicos}. O resumo deve ser  precedido da referência do documento, com exceção do resumo inserido no  próprio documento. (\ldots) As palavras-chave devem figurar logo abaixo do resumo, antecedidas da expressão \textbf{Palavras-chave}:, separadas entre si por ponto e finalizadas também por ponto.

 \textbf{Palavras-chaves}: latex. abntex. editoração de texto.
\end{resumo}

% resumo em inglês
\begin{resumo}[Abstract]
 \begin{otherlanguage*}{english}

   This is the english abstract.

\todo[inline]{fazer tradução do resumo, não utilizar tradução automática}

\todo[inline]{Cuidado com termos que só fazem sentido na língua portuguesa, o texto deve ser ajustado para fazer sentido aos leitores que não conhecem a língua portuguesa}

   \vspace{\onelineskip}

   \noindent 
   \textbf{Keywords}: latex. abntex. text editoration.
 \end{otherlanguage*}
\end{resumo}