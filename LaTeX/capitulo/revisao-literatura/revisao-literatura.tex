% ---
% Capitulo de revisão de literatura
% ---

    Fundado em 1986, o programa de au pair nos \ac{usa} começou como um programa de diplomacia. Desde então, cresceu e se tornou algo muito maior. É uma forma de difundir a tolerância e compreensão intercultural. É uma chance de construir laços com pessoas de todas as esferas da vida. É uma oportunidade dos jovens se tornarem cidadãos globais, modelos e embaixadores de países próximos e distantes \cite{culturalCare2022}.
    
    Au pair é uma expressão da língua francesa que significa ``ao par'' ou ''igual'' e tem como origem a ideia de equilíbrio econômico entre serviços trocados. No mundo todo a expressão au pair é utilizada para definir jovens estrangeiros que participam de um programa de intercâmbio cultural, onde em troca de cuidar das crianças da família recebe uma bolsa de estudos e moradia.
    
    São muitos os tipos de intercâmbios culturais possíveis, algumas alternativas vão desde trabalhos voluntários até estágios remunerados, ou a possibilidade de ser au pair cuidando de idosos. Há também a possibilidade  no mercado de intercâmbio para os mais jovens de cursar o ensino médio em escolas renomadas dos \ac{usa} e ter uma experiência de intercâmbio cultural, esse formato é o famoso \gls{high-school} no \ac{usa} \cite{stbAuPair2022}.

    \section{Experiência au pair}

    Nas plataformas de programas au pairs mais seguras e confiáveis disponíveis no mercado, há lacunas a serem preenchidas em relação principalmente a comunicação e de perfis au pairs dentro deste contexto de intercâmbio.
    
    A construção de uma plataforma que somasse e que contribuísse no processo complexo que o intercâmbio au pair exige partiu de dois sites: \cite{auPair.com2022} e \cite{auPairInAmerica2022}.
    Ambos com mais de 20 anos de experiência no mercado au pair. Porém, outros veículos forneceram novas interpretações do programa de intercâmbio au pair, por exemplo o \cite{bbcNewsBrasil2017} que trouxe uma interpretação do ponto de vista econômico do programa, e traz uma provocação em relação ao custo benefício e a exploração do trabalho. 
    
    Outros aspectos específicos desse tipo de intercâmbio cultural podem ser encontrados e esclarecidos em sites brasileiros consolidados no mercado de intercâmbio, como o \cite{partiuIntercam2022}, \cite{interViagem2022}, que traz dados concretos de custos de programas au pairs, o \cite{culturalCare2022}, com informações gerais, \cite{eurekaFacts2020} com pesquisas de mercado, além das duas plataformas que representam os dois sites norte americanos mais respeitados no mercado, que são \cite{experimCultl2022} e \cite{centralInter2022}. A construção do entendimento da dimensão do significado e da seriedade do intercâmbio cultural com as características au pair foi sendo concebida em torno de diversas referências, e que fundamentalmente compuseram a visão geral do programa.
    
    \section{A importância da internet no mercado au pair}
    
    Há muitos sites na internet que oferecem informações em relação ao programa de intercâmbio au pair. É necessário verificar as fontes para constatar a procedência de seus conteúdos e comprovar se são realmente verdadeiras. Não há outra forma de construir o caminho de intercâmbio cultural au pair que não seja por meio da internet, mesmo as agências trabalhando em torno de alimentar os perfis tanto das au pairs quanto das famílias por meio de seus sites.
   
    O site \cite{auPair.com2022}, referência fundamental para este tipo de intercâmbio deixa claro que há muitas empresas mal-intencionadas e que a candidata ao programa de intercâmbio au pair, deve em qualquer situação entrar em contato com as autoridades legais ou embaixadas do país de destino para confirmar tanto as informações do programa de au pair quanto às informações em relação ao visto de permanência no país.
    Outra importante informação em relação ao visto de permanência é que em algumas situações, há o estímulo para que a candidata utilize outro tipo de visto de permanência que não o de trabalho, o site \cite{auPair.com2022} é categórico quando diz
    
    \begin{citacao}
    Responda NÃO aos que lhe pedem para se tornar seu au pair com um visto de turista de 6 meses ou apenas com um ESTA, que só é válido por até 3 meses! Isto é ilegal! Assim que você atravessar a fronteira dos EUA, você será enviado de volta ao seu país! O programa au pair pode variar de acordo com as regulamentações do país anfitrião. A e o au pair e a família anfitriã devem verificar quais são os requisitos que devem ser cumpridos. Recomendamos que as e os au pairs escolham um país e leiam os requisitos gerais para se tornarem au pair. Se a e o au pair precisar de um visto para o país anfitrião, a embaixada do país anfitrião - localizada no país de residência da ou do au pair - decidirá se a pessoa é elegível para obter o visto. 
    \cite{auPair.com2022}
    \end{citacao}
    
    Mesmo assim, as plataformas digitais ainda são as maiores fontes de informações em relação ao programa de intercâmbio au pair e elas trazem benefícios pela rapidez e facilidade de acesso às informações e pela possível pesquisa por parte das candidatas ao programa au pair em confirmar a idoneidade das empresas divulgadas na internet.
    É interessante destacar que as empresas mais sérias, propõem sempre que um \textbf{contrato} seja estabelecido entre as partes, e há alguns pontos de atenção fundamentais a serem pesquisadas pelas candidatas em relação a isso, segundo o site \cite{auPair.com2022} o fundamental é que este contrato deve conter, como demonstrado abaixo:

    \begin{quadro}[H]
    \centering\footnotesize
    \caption{Estrutura de contrato au pair}
    \label{quadro-contrato-aupair}
        \begin{tabular}{|p{0.32\linewidth} | p{0.62\linewidth} |}  \hline
        \thead{Descrição} & \thead{Detalhes} \\
        \hline
        Duração da estadia & Quanto tempo a estadia durará?
        \\
        \hline
        Data de início e fim do contrato & Quando a estadia da/do au pair vai começar? Quando vai terminar? 
        \\
        \hline
        Pagamento mensal / semanal & Quanto a au pair ganhará em troca do serviço fornecido?
        \\
        \hline
        Férias & Quantos dias livres a/o au pair terá e quando?  
        \\
        \hline
        Acomodações e refeições & Onde vai morar a au pair? Descrever o quarto da/do au pair, regras gerais, etc. 
        \\
        \hline
        Horas de trabalho & Quantas horas a au pair deve trabalhar? Qual será o horário de trabalho?  
        \\
        \hline
        Despesas de transporte no exterior & Quem vai cobrir as despesas de transporte no país de acolhimento? A au pair terá o direito de usar o carro da família? A au pair receberá um cartão mensal de transporte?  
        \\
        \hline
        Seguro & Quem será responsável pelo seguro médico da/do au pair? A au pair terá algum seguro extra (responsabilidade, acidente, etc.)?  
        \\
        \hline
        Responsabilidades de au pair & Quais serão os deveres da/ do au pair durante sua estadia?  
        \\
        \hline
        Despesas de viagem & Quem vai pagar as despesas da viagem da au pair para chegar ao País Anfitrião?  
        \\
        \hline
        Cancelamento de contrato & Qual é o prazo de aviso prévio em caso de cancelamento do contrato?
        \\
        \hline
        Custos do visto & Quem vai pagar pelas despesas do visto?
        \\
        \hline
        \end{tabular}
    \fonte{Site: \cite{auPair.com2022}}
    \end{quadro}
    
    Com essas plataformas virtuais verificadas e avaliadas pelos envolvidos no processo de candidatura au pair, com uma pesquisa intensa em relação às vagas, as ferramentas e possibilidades de intercâmbio, uma experiência au pair de intercâmbio cultural positiva é possível e são muitos os relatos e as histórias de sucesso divulgadas nas redes, algumas com extrema sinceridade.
    
    \begin{citacao}
    Ficamos muito felizes com a plataforma vindo de uma experiência negativa com outra plataforma (apesar de termos pago, não conseguimos encontrar nenhuma au pair disponível). Muito úteis são as funções que mostram o nível de atividade da au pair. Esta foi nossa primeira experiência com um aupair que terminou há uma semana. A experiência foi positiva, embora esperássemos mais entusiasmo e participação em atividades familiares por parte da au pair. Se repetirmos a experiência no futuro, definiremos melhor as regras da casa, especialmente no que diz respeito à limpeza do uso exclusivo da au pair.
    \cite{auPair.com2022}
    \end{citacao}
    
    Portanto é necessário destacar a necessidade de ampliação da oferta de divulgação de perfis au pairs para aumentar as chances de soluções para que encontros entre famílias anfitriãs e au pairs aconteçam com maior frequência, segurança e conforto.