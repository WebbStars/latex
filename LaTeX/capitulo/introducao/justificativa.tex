\section{Justificativa}

    O programa de intercâmbio au pair é uma possibilidade de experiência internacional de baixo custo e ampla vivência cultural, social e acadêmica, pois tem no seu desenho original, além da convivência com uma família anfitriã, a possibilidade de aprender a língua do país e há ainda a oportunidade de frequentar um espaço educativo, o que amplia muito a experiência como um todo. 
    
    Por isso, o intercâmbio au pair tem sido considerado vantajoso por muitos jovens que buscam ter experiências culturais no exterior. Além do programa compreender benefícios como custeio de passagens, cursos de idiomas, acomodação, alimentação durante toda a estadia e salários por funções desempenhadas, ainda deve ser considerada a experiência cultural e social que o programa proporciona. 
    
    Diante de uma pesquisa realizada pelo
    \cite{eurekaFacts2020}, que é um site de soluções inteligentes de pesquisa, foi possível mensurar o impacto do programa, pois percebeu-se que os objetivos desejados entre os sujeitos envolvidos têm tido um nível satisfatório.
    
    Foram entrevistados 10.881 participantes de programas au pair e 6.452 famílias anfitriãs, onde constatou-se que:
    
    \begin{itemize}
        \item 97\% das au pairs entrevistadas obtiveram uma melhor compreensão da cultura americana (tradições e costumes);
        \item 90\% das au pairs entrevistadas consideraram excelente (60\%) ou bom (30\%) a experiência que tiveram;
        \item 88\% das au pairs entrevistadas dizem ter melhor compreensão dos valores americanos (liberdade e independência); 
        \item 86\% das famílias anfitriãs entrevistadas dizem manter contato com as participantes mesmo depois da saída do país e fim do programa.
    \end{itemize}
    
    Segundo o site \cite{valorIveste2022}, a pesquisa aponta que o mercado pós crise sanitária da \ac{covid} passa a ter nova demanda de interessados neste programa de intercâmbio. 
    
    \begin{citacao}
       Depois de quase dois anos com viagens canceladas, planos adiados e quebras de contrato, as empresas especializadas em intercâmbio, imigração, vistos e green card começam a ver o mercado voltar a crescer, retomando os patamares de antes da pandemia. (...) “Para 2022 temos uma grande expectativa e projetamos um crescimento bem mais significativo que o do ano passado, principalmente nos serviços de visto e green card para os brasileiros que já estão nos EUA e não pensam em voltar; tanto para aqueles que querem recomeçar no exterior”, comenta Arleth Bandera, a \ac{ceo} da Eagle Intercâmbio.”
       \cite{valorIveste2022}
    \end{citacao}
    
    Diante deste contexto, o mercado de intercâmbio au pair ganha espaço significativamente, e o desenvolvimento de uma plataforma que facilite, amplie e valorize a conexão entre au pairs, famílias anfitriãs e agências expande as oportunidades.
    Ainda com a expansão e crescimento das relações conectadas por meio da internet e o imenso fluxo de pessoas usufruindo e consumindo ferramentas virtuais, surge a necessidade do desenvolvimento de um sistema interligado, alinhado às demandas advindas dos programas de intercâmbio au pair para contribuir e construir novas relações internacionais.
    
    O intercâmbio cultural é um dos desejos mais cobiçados pelos jovens e as ofertas disponíveis no mercado nem sempre são claras e suficientemente simples para proporcionar que este desejo se realize, por conta especialmente do processo de análise e dificuldade de \gls{match} é que se tornar au pair pode ficar mais difícil, a plataforma
    \href{https://aupamatch.pages.dev/home}{AupaMatch} vem para contribuir neste aspecto no sentido de ampliar as ofertas e possibilidades para que o encontro aconteça.
