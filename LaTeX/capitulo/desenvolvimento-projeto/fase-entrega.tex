    \section{Entregáveis por Fase de Entrega}
    O Projeto \textbf{AupaMatch} foi dividido em três grandes fases de entregas, exatamente para corresponder a demanda de produção e validação da aplicação web. As três fases de entregas são: a primeira é a \textbf{Proof of Concept} (\ac{poc}), a segunda o \textbf{Minimum Viable Product} (\ac{mvp}) e a última o \textbf{Produto Acabado}. Nas seções seguintes há uma descrição de cada etapa de entrega e o significado desses conceitos.

    \subsection{PoC}
    A \ac{poc} é uma “prova de conceito”, ela serve para que possa ser realizada uma experiência no mercado, porém, sem muita exposição da aplicação web. É a possibilidade de verificar erros e acertos de um software e, assim, se tornar uma evidência da possibilidade de se obter bons resultados no funcionamento.
    Segundo o site do Sebrae a definição do conceito de uma \ac{poc} pode ser descrito como.
    
    \begin{citacao}
        Uma \ac{poc} (Proof of Concept) é a evidência documentada de que um software pode ser bem-sucedido. Ao fazer uma \ac{poc}, é possível identificar erros técnicos que possam interferir no funcionamento e nos resultados esperados. Além disso, a prova de conceito permite a solicitação de \gls{feedbacks} internos e externos. Assim, os testes são realizados sem muita exposição e permite-se a correção de erros e implementação de melhorias.
        \cite{sebrae2018}
    \end{citacao}
    
    A \ac{poc} é a primeira fase das três principais entregas do projeto. Consiste em um protótipo funcional do produto que será entregue. Este protótipo deve ser capaz de demonstrar a funcionalidade do produto e, dessa forma, permitir que seja possível avaliar sua utilidade. Tem-se como objetivo final de uma \ac{poc}, fornecer uma visão clara do produto que está sendo desenvolvido e permitir que seja possível avaliar sua utilidade. Este é um passo importante para garantir que o produto atenda às necessidades e evite futuras rejeições, portanto a \ac{poc} é um conceito que se tornou hoje fundamental no mercado da tecnologia.
        
    \begin{citacao}
        A metodologia de \gls{proof-of-concept} (\ac{poc}) se tornou parte importante do dia a dia de empresas, especialmente das \gls{startups}. Como negócios inovadores e preocupados em lançar produtos que, de fato, resolvam problemas dos seus clientes, é quase indispensável que essas empresas utilizem o \ac{poc} como metodologia em seus processos. \gls{proof-of-concept} (\ac{poc}) é a validação de uma ideia de negócio. Consiste em “tirar a prova” ou validar aquele conceito no mercado. Em outras palavras, trata-se de saber que aquela ideia de produto ou serviço vai encontrar clientes e uma audiência interessada em adquiri-la. Sendo assim, o \ac{proof-of-concept} normalmente é utilizado por empresas pequenas ou \gls{startups} com uma ideia de solução para determinado problema. Naturalmente, é assim que muitas \gls{startups} surgem — elas identificam um problema ou uma dor de determinado grupo de pessoas e criam soluções para resolvê-los.
        \cite{ranDon2022}
    \end{citacao}
    
    \subsection{MVP}
    O \ac{mvp} é a segunda das três principais entregas do projeto.
    Do inglês o “\ac{minimum-viable-product}” ou em português Produto Mínimo Viável, trata-se do conjunto mínimo dos elementos necessários para que o produto possa ser levado para produção e possa ser utilizado num cenário real. Seria o momento de validar as ideias, como o próprio nome sugere, esta etapa se refere ao mínimo que é preciso para ter algo viável na produção da aplicação web. 
    É importante definir com a equipe de produção o \ac{mvp} esperado, já que  o objetivo de um \ac{mvp} é fornecer um produto que esteja pronto para o mercado. Uma descrição interessante dessa importante fase poderia ser:
    
    \begin{citacao}
        A ideia por trás do \ac{mvp} é desenvolver uma versão de teste do seu projeto, com o mínimo de investimento financeiro e de tempo, mas capaz de entregar os mesmos valores do produto finalizado. Dessa forma, a ideia pode ser testada e, se aprovada, toda a estrutura necessária para o desenvolvimento é aplicada.
        \cite{rockContent2022}
    \end{citacao}
    
    Portanto, \ac{mvp} Consiste em produtos funcionais que atendam às necessidades esperadas. Este é um passo fundamental para avaliar se o produto está pronto e sem falhas na sua execução além de ser o momento de validação para sua produção final. Outra referência interessante para entender a importância da fase \ac{mvp} é
    
    \begin{citacao}
        É um conjunto de testes primários feitos para validar a viabilidade do negócio. São diversas experimentações práticas que serão desenvolvidas levando o produto a um seleto grupo de clientes… mas não é o produto final! Estamos falando em um produto com o mínimo de recursos possíveis, desde que (em sua totalidade) estes mantenham sua função de solução ao problema para o qual foi criado. Essa técnica de 3 letras ajudou gigantes como \gls{facebook}, \gls{apple} e \gls{dropbox} a se consolidarem em seus segmentos, sem gastarem horrores nos períodos iniciais.
        \cite{endeavor2022}
    \end{citacao}
    

    Consiste em produtos funcionais que atendam às necessidades esperadas. O objetivo de um \ac{mvp} é fornecer um produto que esteja pronto para o mercado. Este é um passo fundamental para avaliar se o produto está pronto e sem falhas na sua execução além de ser o momento de validação para sua produção final.
    
    \subsection{Produto Acabado}
    O Produto Acabado é a terceira e última fase das três principais entregas do projeto. Consiste em produtos finais que atendam às necessidades e expectativas. O produto deve ser capaz de entrar no mercado e gerar receita.