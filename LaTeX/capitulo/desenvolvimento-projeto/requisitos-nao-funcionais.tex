\subsubsection{Requisitos Não Funcionais}
Com o propósito de especificar as características que o sistema deve conter, define-se os \gls{requisitos-nao-funcionais} que não estão associadas as funcionalidades específicas, sendo que algumas delas não estão perceptíveis para qualquer usuário. 

Aspecto de desempenho da aplicação mediante aos critérios de requisitos mínimos que foram provisionados para o seu funcionamento devendo ser previamente mensurável, a disponibilidade de acesso ao conteúdo das informações da aplicação, quais conteúdos cada usuário pode acessar, portabilidade por meio de fatores do sistema ser adaptável a mudanças, outro fator é a usabilidade da qual o sistema utiliza métodos de controle de acesso, apresentação, estética da interface do usuário, operação e proteção contra erros.

Os requisitos funcionais foram mensurados pela equipe do projeto, o qual buscou enfatizar os controles técnicos visando mitigar riscos relacionados à segurança e garantir escalabilidade e usabilidade, sendo definidos no \autoref{requisitos-nao-funcionais}.
    
    \begin{enumerate}
        \begin{quadro}[H]
        \centering\footnotesize
        \footnotesize
        \caption{Requisitos Não Funcionais}
        \label{requisitos-nao-funcionais}
            \begin{tabular}{|p{0.15\linewidth} | p{0.15\linewidth} | p{0.15\linewidth} | p{0.35\linewidth} |}  \hline
            \multicolumn{1}{|c|}{\textbf{Código do Requisito}} &
              \multicolumn{1}{c|}{\textbf{Módulo}} &
              \multicolumn{1}{c|}{\textbf{Nome}} &
              \multicolumn{1}{c|}{\textbf{Descrição}} \\ \hline
            REQNF01 &
              Usabilidade &
              Usabilidade Sistêmica &
              A aplicação deve conter uma interface desenvolvida para atender a utilização do usuário de forma responsiva \\ \hline
            REQNF02 &
              Performance &
              Escalabilidade de Armazenamento &
              A aplicação deve ser escalável com propósito de suportar aumento e redução de capacidade de armazenamento de dados de acordo com a demanda \\ \hline
            REQNF03 &
              Disponibilidade & Disponibilidade de Dados &
              O provedor dos recursos de hospedagem deve garantir disponibilidade mínima de 95\% ao mês \\ \hline
            REQNF04 &
              Segurança &
              Protocolo de Segurança de dados &
              A aplicação deve conter controle de criptografia TLS ou SSL.  \\ \hline
            REQNF05 &
              Logs &
              Armazenamento de logs &
              A aplicação deve conter logs de transações. \\ \hline
            \end{tabular}
        \fonte{Autores}
        \end{quadro}
    \end{enumerate}
