\subsubsection{Histórias de Usuário}

Em um ambiente ágil, as \ac{histórias de usuário} é um instrumento de escrita utilizado no processo de levantamento de requisitos para descrever a especificação de uma funcionalidade do software, afirmam Beck e Fowler (2001). 

As \ac{histórias de usuário} normalmente seguem o padrão de papel-função-para, onde:\\
•	Como um <tipo de usuário>;\\
•	Gostaria de <algum recurso>;\\
•	Para <alguma razão>.

Para a criação das histórias de usuário utilizaremos o critério de aceitação \ac{bdd}, que significa desenvolvimento orientado por comportamento que se divide em 5 passos, sendo eles: foco em cenário, especificação para o cenário, especificação das unidades, especificação da unidade passar e refatore. 

As \ac{histórias de usuário} que formam nosso \ac{product-backlog} são:

• REQF001 – Usuário (Gerenciador de usuário)
\\ - Quero criar, editar ou excluir uma conta na aplicação;
\\ - Como usuário au pair, família ou agência quero poder criar, editar e excluir a minha conta para ter autonomia sobre minha conta.

• REQF002 – Login (Identificação e autenticação de usuário)
\\ - Quero me autenticar na aplicação;
\\ - Como usuário au pair, família ou agência gostaria de me logar na aplicação para ter acesso às funcionalidades que o sistema disponibiliza.

• REQF003 – Perfil (Definição de perfil de usuário)
\\ - Quero definir meu perfil; 
\\ - Como usuário au pair, família ou agência gostaria de definir meu perfil para ter acesso às funcionalidades da aplicação. 

• REQF004 – Vaga (Gerenciador de anúncios de vaga)
\\ - Quero criar, editar ou excluir anúncios de vagas;
\\ - Como usuário família ou agência gostaria de criar, editar ou excluir anúncios de vagas para obter candidaturas. 

• REQF005 – Candidatar (Aplicar candidaturas)
\\ - Quero me candidatar nas vagas em aberto;
\\ - Como usuário au pair gostaria de me candidatar nas vagas em aberto para o processo de seleção.

• REQF006 – Favoritos (Salvar vagas em favoritos)
\\ - Quero salvar vagas em favoritos;
\\ - Como usuário au pair, família ou agência gostaria de selecionar as vagas em aberto e salvá-las na lista de favoritos para visualizar mais facilmente as vagas que selecionei posteriormente.

• REQF007 – Consolidado de vagas (Envio do consolidado de novas vagas)
\\ - Quero ter a opção de receber o consolidado das novas vagas dos últimos 7 dias;
\\ - Como usuário au pair gostaria de ter a opção de receber e-mail de consolidação de novas vagas publicadas nos últimos 7 dias para me informar sobre as mesmas. 

• REQF008 – Filtro de busca (Filtrar perfis com as melhores avaliações)
\\ - Quero filtrar as agências e famílias com as melhores avaliações.
\\ - Como usuário au pair, família ou agência gostaria de filtrar perfis para encontrar aqueles com as melhores avaliações.

• REQF009 – Avaliação de perfil (Os perfis podem avaliar outros perfis)
\\ - Quero avaliar perfis;
\\-  Como usuário au pair, família ou agência gostaria de opinar sobre o serviço que me foi oferecido para determinar a eficiência e qualidade do serviço prestado ou recebido. 

• REQF010 – Priorização e vagas (Vagas com prioridade de visualização)
\\ - Quero publicar vagas com prioridade de visualização;
\\ - Como usuário agência ou família gostaria de publicar vagas com prioridade de visualização para obter mais atenção dos perfis candidatos para que a vaga seja ocupada em menor tempo.  



