\subsubsection{Requisitos Funcionais}
Durante a etapa de aprofundamento das funcionalidades e de como os recursos do sistema deve se comportar, o projeto requer um maior aprofundamento nas especificações de requisitos para etapa de desenvolvimento, por meio da definição dos \gls{requisitos-funcionais}, a fim de que atenda às necessidades ou expectativas do usuário por meio do comportamento da detecção da execução de funções os quais aparecem com as entradas e saídas, mostrando resultados para a tela do usuário.

A equipe do projeto identificou os requisitos funcionais, os quais são premissas durante o desenvolvimento do projeto, sendo específicos sobre o que o sistema deve entregar, mensuráveis para garantir indicadores de execução, alcançáveis atendendo os prazos do projeto, relevantes para garantir os objetivos do negócio, limitados em seu escopo para possibilitar o acompanhamento do progresso. 
Os requisitos funcionais do projeto foram catalogados e descritos no \autoref{requisitos-funcionais}.



\begin{enumerate}
    \begin{quadro}[H]
    \footnotesize
    \caption{Requisitos Funcionais}
    \label{requisitos-funcionais}
        \begin{tabular}{|p{0.10\linewidth} | p{0.11\linewidth} | p{0.2\linewidth} | p{0.35\linewidth} |}  \hline
          \multicolumn{1}{|c|}{\textbf{Código do Requisito}} &
          \multicolumn{1}{c|}{\textbf{Módulo}} &
          \multicolumn{1}{c|}{\textbf{Nome}} &
          \multicolumn{1}{c|}{\textbf{Descrição}} \\ \hline
        REQF01 & Usuário.  &
            Gerenciador de Usuário.                & 
            A aplicação deve permitir que o usuário tenha permissão criar, editar e excluir um usuário. \\  \hline
        REQF02 & Login.  &
            Identificar e autenticar com usuário.  &
            A aplicação deve permitir que o usuário realize o acesso em sua conta.                      \\ \hline
        REQF03 & Perfil.   &
            Definir perfil de usuário             & 
            A aplicação deve permitir que o usuário defina um perfil (Au pair, família ou agência).     \\ \hline
        REQF04 & Vaga. &
            Gerenciador de anúncio de vaga. 
            & Família ou agência devem poder criar, editar, excluir vagas.                                \\ \hline
        REQF05 & Candidatar. &
        Aplicar candidatura em vagas. &
        Au pair pode se candidatar a uma ou mais vagas.                                      \\\hline
        REQF06 &
          Favoritos. &
          Salvar vagas em favoritos. &
          Au pair, família e agência podem salvar todas as vagas em aberto na lista de favoritos. \\ \hline
        REQF07 &
          Consolidado de vagas. &
          Envio do consolidado de novas vagas. &
          Au Pair tem opção de receber e-mail de consolidado de novas vagas publicadas dos últimos 7 dias.  \\ \hline
        REQF08 &
          Filtro de busca. &
          Filtrar perfis de  au pairs, famílias e agências com as melhores avaliações & 
          Au pair, famílias e agências devem poder filtrar os perfis melhores avaliados. \\ \hline
        REQF09 & 
            Avaliação de perfil. & 
            Os perfis podem avaliar outros perfis & Os perfis devem poder avaliar o perfil de outros usuários.                                  \\ \hline
        REQF10 &
          Priorização de vagas. &
          Vagas com prioridade de visualização. &
          A agência e família devem poder publicar vagas com prioridade de visualização. \\ \hline
        \end{tabular}
    \fonte{Autores}
    \end{quadro}
\end{enumerate}
