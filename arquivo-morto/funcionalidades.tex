\section{Funcionalidades}
As funcionalidades de um sistema são o início para a estruturação do projeto. Conforme o \autoref{funcionalidades-descricoes}, são apresentado as funcionalidades da plataforma e suas descrições.

\begin{quadro}[H]
	\centering\footnotesize
	\caption{Funcionalidades e Descrições}
	    \label{funcionalidades-descricoes}
	   \resizebox {1 \textwidth }{!}{
            \begin{tabular}{|l|p{8cm}|}
            \hline
            \thead{Funcionalidade} & \thead{Descrição} \\ \hline
            Cadastro de usuário & Todos os usuários devem se cadastrar com e-mail e senha e devem definir um perfil (Au pair, família ou agência).        
            \\ \hline
            Publicador de vaga para au pair & agência ou família podem publicar uma vaga para inscrição de au Pair.      
            \\ \hline
            Filtro de busca de vagas e perfis & O filtro de busca tem a finalidade de buscar as informações que os usuários estão procurando.
            \\ \hline
            Agenda para disponibilidade de trabalho & Au pair podem informar sua disponibilidade para trabalho por meio de agenda.
            \\ \hline
            Alerta de visualização de perfil & Agência, família e au pair recebem alertas de visualização de perfil por e-mail.
              \\ \hline
            Consolidado de vagas & au pairs recebem e-mail de consolidado de novas vagas publicadas dos últimos 7 dias.  
            \\ \hline
            Exibição de visitas de vagas &  É exibido o número total de visitas à vaga publicada.
            \\ \hline
            Entrar em contato com au pair e família & Au pair e família possuem opção de entrar em contato por meio de caixa de mensagem que ao preencher deve enviar um e-mail para o próprietario do perfil.
            \\ \hline
            Salvar vagas em favoritos & Permite salvar o perfil visualizado em favoritos.     
            \\ \hline 
            Avaliação de au pair, famílias e agências & Deve ser gerado um score para um au pair, família e agência que receberam uma avaliação por nota, os melhores avaliados ganham destaque nos filtros de busca de perfis.
            \\ \hline
            \end{tabular}
        }
\fonte{Autores}
\end{quadro}
O \autoref{funcionalidades-usuarios} exibe cada funcionalidade para determinado perfil. Os perfis da plataforma são os perfis família, au pair e agência.

\begin{quadro}[H]
	\centering\footnotesize
        \caption{Funcionalidades e Perfis}
        \label{funcionalidades-usuarios}
            \begin{tabular}{|l|c|c|c|}
                \hline
                \thead{Funcionalidade}                      & \thead{Família} & \thead{Au pair} & \thead{Agência} \\ \hline
                Cadastro de usuário                                    & x       & x       & x       \\ \hline
                Filtro de busca de vagas e perfis                                        & x       & x       & x       \\ \hline
                Agenda para disponibilidade de trabalho                             & x       & x       & x       \\ \hline
                Alerta de visualização de perfil                          & x       & x       & x       \\ \hline
                Consolidado de vagas                                 & x       & x       & x       \\ \hline
                Exibição de visitas de vagas                      & x       &        & x       \\ \hline
                Entrar em contato com au pair e família                           & x       & x       & x       \\ \hline
                Salvar vagas favoritos                      &        & x       &       \\ \hline
                Avaliação do perfil   & x       & x       & x       \\ \hline
                Apresentação das melhores vagas avaliadas   &        & x       &        \\ \hline
                Apresentação das melhores vagas avaliadas   &        & x       &        \\ \hline
                Avaliação de vagas de famílias e au pairs   & x       & x       & x       \\ \hline
                Publicador de vaga para au pair                         & x       &        & x       \\ \hline
            \end{tabular}
\fonte{Autores}
\end{quadro}